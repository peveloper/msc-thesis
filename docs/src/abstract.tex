\addchap*{\centering Abstract}

In the last recent years streaming services such as Netflix, Youtube, Amazon
Prime Video, Hulu and others, have become the main source for video content
delivery to the public. With the effort of private companies and of the AOM
consortium, various coding formats and streaming techinques have been refined
and have gain popularity. \textit{Adaptive Bitrate Streaming}, between others,
enables high quality streaming of media content over HTTP, and represents
nowadays the industry's standard. 

DASH \textit{Dynamic Adaptive Streaming over HTTP} is an instance of Adaptive
Bitrate Streaming originally developed by MPEG. In DASH each media file gets
encoded at multiple bitrates, which are then partitioned into smaller segments
and delivered to the user over HTTP.  Netflix's use of DASH services is no
mistery, indeed it is already five years that each title on Netflix sits with
its own different bitrate copies on a CDN, waiting to be served to clients in a
particular area of the planet. \cite{per-title-encoding}

Despite a recent upgrade in Netflix infrastructure to provide HTTPS encryption
of each video stream, research shows that the privacy of the end user is at
risk, more precisely there exist techniques to identify the content the client
is playing as Reed et Al. \cite{netflix-real-time} have shown. They make use of
adudump \cite{adudump} a command live program that uses TCP sequence and ACKS
to infer the sizes of application data unit \emph{ADUs} transferred over each
TCP connection. Our approach is mainly based on their work, with few
differences highlighted in Section X.

%TODO refine it and add results
