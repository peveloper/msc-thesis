\addchap*{\centering Abstract}

In the last recent years streaming services such as Netflix, Youtube, Amazon
Prime Video, Hulu and others, have become the main source for video content
delivery to the public. With the effort of private companies and of the AOM
consortium \cite{aom}, various coding formats and streaming techinques have
been refined and have gained popularity. \textit{Adaptive Bitrate Streaming},
between others, enables high quality streaming of media content over HTTP, and
represents nowadays the industry's standard. 

DASH \textit{Dynamic Adaptive Streaming over HTTP} is an instance of Adaptive
Bitrate Streaming originally developed by MPEG. In DASH each media file gets
encoded at multiple bitrates, which are then partitioned into smaller segments
and delivered to the user over HTTP.  Netflix's use of DASH services is no
mistery, indeed it is already five years that each title on Netflix sits with
its own different bitrate copies on a CDN, waiting to be served to clients in a
particular area of the planet. \cite{per-title-encoding}

Reed et Al. \cite{netflix-real-time} have shwon how, despite a recent upgrade
in Netflix infrastructure to provide HTTPS encryption to video traffic, it is
possible to recover unique fingerprints for each title, due to the adoption
throughout the entire Netflix library, of \emph{per-title encoding}. They make
use of adudump \cite{adudump} a command-line program that can run on passive
TAP device \cite{tap} or on a live network interface, and uses TCP and ACKS
sequences to infer the sizes of application data unit \emph{ADUs} transferred
over each TCP connection. 

Our approach reiterates parts of Reed et Al.'s work, but cannot rely on their
every assumption and discovery, due to constant changes that Netflix is brining
to their enconding and streaming algorithms. Moreover our intent is focused on
finding out if, by analyzing coarse-grained traffic data, we are able identify
a video based on its \emph{bitrate-ladder}.

%TODO refine it and add results
