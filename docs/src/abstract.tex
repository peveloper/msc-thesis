\addchap*{\centering Abstract}

In the last recent years streaming services such as Netflix, Youtube, Amazon
Prime Video, Hulu and others, have become the main source for video content
delivery to the public. With the effort of private companies and of the AOM
consortium \cite{aom}, various coding formats and streaming techinques have
been refined and have been vastely adopted by the industry.  \emph{Variable
Bitrate Encoding} and \emph{Adaptive Bitrate Streaming}, between others, are
two techniques that increasingly often get coupled together to improve high
quality streaming of media content over HTTP.

Netflix relies on first encoding a title at a \emph{variable} bitrate, and
streaming it using DASH \textit{Dynamic Adaptive Streaming over HTTP}, an
instance of Adaptive Bitrate Streaming originally developed by MPEG. In DASH
each media file gets encoded at multiple bitrates, which are then partitioned
into smaller segments and delivered to the user over HTTP. As of today, Netflix
makes extensive use of DASH, by encoding each title in their catalogue at
different quality levels, so that the user is served tailored-size content
depending on network conditions, and its device capabilities
\cite{per-title-encoding}.

DASH and \emph{variable-bitrate-encoding}, have proven to be leaking
potentially harmful data. Reed et Al. \cite{netflix-real-time} have shown, how
despite a recent upgrade in Netflix infrastructure to provide HTTPS encryption
to video traffic, it is possible to recover unique \emph{fingerprints} for each
title, and match them against future traffic traces. In their study, they make
use of adudump \cite{adudump} a command-line program that can run on passive
TAP devices \cite{tap} or on a live network interface, and uses TCP and ACKs
sequences to infer the size of application data units \emph{ADUs} transferred
over each TCP connection in real time.

Our approach reiterates parts of Reed et Al.'s work, but cannot rely on their
every assumption and discovery, due to constant changes that Netflix is
integrating into their enconding and streaming algorithms. Moreover our intent
is focused on finding out if, by analyzing coarse-grained traffic data, we are
able to identify a video based on its \emph{bitrate-ladder}.

\todo{refine it and add results}
