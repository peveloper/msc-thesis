\addchap*{\centering Abstract}

In recent years, tools for program verification have made significant progress
and are becoming more widely used. Yet, they still lack effective facilities to
allow investigating and understanding verification errors, especially when the
input program makes use of more advanced language features, such as quantified
permissions in Viper.

We think the most effective way to help a user in identifying the source of
a verification error is by employing a visual debugging approach, therefore we
want to provide a technique to automatically produce small, visual
counterexamples based on the information provided by a symbolic execution
engine.

We conducted a feasibility study to understand whether we could effectively
generate counterexamples via bounded modeling with Alloy, a language and
analyser for software modeling. The main idea behind this methodology is that
Alloy, because it performs a bounded search, is not affected by the problems
involving quantifier instantiation, and is therefore able to provide complete
concrete models in situations where the SMT solver would not.

In our technique, we encode the information about a symbolic execution state
into a model and use Alloy to generate instances of it. These instances
can then be used to build a visual diagram of the program's state. When the
state being modeled is one where a verification error occurred, and we
additionally encode the last failed query performed to the SMT solver, then the
instances Alloy generates are counterexamples to the failed verification.

As part of our feasibility study, we have implemented a subset of the technique
described in this thesis into a tool, integrated with the Viper IDE. This proof
of concept for a debugger demonstrates that our approach for visualizing
counterexamples to verification failures in the context of symbolic execution is
a feasible one and is worth exploring in more detail.
