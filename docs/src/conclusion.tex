\chapter{Conclusions}\label{sec:conclusion}

We now analyze and discuss our system's strengths and weaknesses, arguing on
the conditions required to successfully perform the attack, and eventually,
present our perspective on possible future research.

\section{Contribution}

In \Cref{sec:results} we have given both quantitative and visual evidence of the
results achieved with our methodology. We now extend the analysis by abstracting
the context to a more realistic scenario. 

Throughout the description of our approach, we have ignore potential factors
that may be source of various types of errors when carrying out the attack in a
real scenario (\emph{e.g.} a public WLAN network). 

We have presented a revisited version of \cite{netflix-real-time} that, by
watching 4 minutes of video playback for a limited number of Netflix titles
(100), in stable network conditions, can faithfully reconstruct bitrate ladders
of movies encoded with an average bitrate greater than 1Mbps across 13
different enforced bandwidth levels. The 100 reconstructed bitrate ladders are
uniquely identifiable by their average bitrate, standard deviation, or median.

Our system identifies with approximately 86\% of accuracy the same set of
video titles across 8 \emph{unseen} bandwidth levels not previously captured in the
database. We have tested this behavior for several window sizes and conclude
that for majority of the titles, the average bitrate of, up to 20 subsequent
ADUs, represents a robust feature to reveal the content being streamed.

\subsection{Metrics}

\section{Benefits}

\todo{finish}

\section{Shortcomings and Feasibility}

\todo{finish}

The presence of various users in the network, for instance, may potentially
impact the quality of our resconstruction, especially if the content being
streamed is encoded at a low bitrate, or if the network bandwidth is unstable
(could cause the Netflix buffering algorithm to jump over different bitrate
levels during a 4 minute capture. 

\section{Future Work}

\todo{finish}
